\documentclass{report} 
\usepackage[T1]{fontenc}
\usepackage{amsmath,amsfonts,amsthm} 
\usepackage[utf8]{inputenc}
\usepackage[latin1]{inputenc} 
\usepackage{lmodern}  
\usepackage{graphicx}
\usepackage{hyperref}
\usepackage[francais]{babel}

\begin{document}
\thispagestyle{empty} 
\includegraphics[width=0.25\textwidth]{ensias.jpg}\\\vfill 
\begin{center}
\Huge \textsc\textbf{cahier de charge}\\\vfill

\textbf{Sujet :\\ Application web e-learning}\normalsize \\\vfill
                                                                  
\begin{tabular}{ll}
\hline\hline\\
Réalisé par: & Encadré par:\\
KHADROUF Oumaima & Mr. NAFIL Khalid\\ 
LARHOUTI Meriem\\
\hline\hline\\                                     
\end{tabular}\\\vfill
\end{center}      


Année universitaire : 2017/2018

\newpage
\tableofcontents
\newpage
\section{Contexte et définition du problème}
L'internet a devenu le réseau de communication le plus important du monde. En effet, c'est grâce aux nouvelles technologies en informatique qu'on a pu changer nos modes de travail à savoir l'enseignement, le commerce, la santé, etc.\\
L'introduction du numérique a entraîné une transformation très profonde des manières d'enseigner. Dans ce cadre ce projet consistera à réaliser un site web qui permettra une formation en ligne.\\
Problématique:\\
La plupart des étudiants marocains en études supérieures trouvent des difficultés à suivre L'enseignement académique, une raison pour laquelle la quête d'une bonne formation devient essentielle. Cependant, cette recherche va obliger l'étudiant à se déplacer, par conséquent il va perdre du temps et d'argent. Alors, comment l'étudiant peut bénéficier d'une bonne formation en ligne ? 
 
\section{Objectif}
La plateforme sera un service de e-learning qui permettra à l'étudiant de bénéficier d'une formation sans se déplacer. L'application va offrir un large choix à l'étudiant entre des cours académique afin d'améliorer ses connaissances et ses notes. 

\section{Périmètre}
La plateforme vise les élèves en études supérieures, auxquelles un ordinateur et une connexion internet sont accessibles. Le site aidera ces étudiants à apprendre des concepts difficiles et à obtenir des meilleures notes en offrant des formations académiques réalisées par des enseignants professionnels. 

\section{Parties prenantes }
Administrateur : celui qui gère les comptes des étudiants et des professeurs ainsi qu'il s'occupe de la gestion de la facturation
\\
Étudiant : celui qui bénéficiera des formations offertes par la plateforme. \\
Professeur : celui qui déposera son cour dans l'application. \\

\section{Descriptions des besoins }

\subsection{Besoins fonctionnels}

Les besoins fonctionnels ou besoin métiers représentent les actions que le système doit
exécuter, il ne devient opérationnel que s'il les satisfait.\\
Cette application doit couvrir principalement les besoins fonctionnels suivants : \\
-Gestion des comptes :\\
	Ajout, suppression de comptes\\
	Définir les types de comptes (simple, fidèle..).
\\
-Gestion de facturation :\\
	Calculer la facture des comptes. \\
	Définir le type d'abonnement des comptes.
\\	
-Gestion des cours:\\
	Ajout, suppression, affichage et modification de cours\\
	
\subsection{Besoins non fonctionnels}
Ce sont des exigences qui ne concernent pas spécifiquement le comportement du système
mais plutôt identifient des contraintes internes et externes du système.\\
 Les principaux besoins non fonctionnels de notre application ce résument dans les points
suivants :\\
-L'interface doit être ergonomique, conviviale et facile à utiliser.\\
-Le système doit être sécurisé.\\
-L'application doit contenir une boite de discussion.



\end{document}